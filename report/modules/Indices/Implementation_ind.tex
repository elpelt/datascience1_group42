All indices are implemented as a class function for the class \textit{Indices}. The code is part of the scikit-library \cite{scikitlearn} and expects two inputs for external cluster validation methods and three inputs for the internal cluster validation method. The input contains two arrays, namely the computed cluster labels and the expected cluster labels from the original data. For the internal index, the input is expected to be the calculated labels, the original data points, and a metric to calculate the distances (default = selected metric for cluster calculation). The output is a numerical value between 0 and 1 for Homogeneity Score and Completeness Score. For the \acrfull{ari} and the internal index Silhouette Score a value between -1 and 1 is calculated. \acrfull{ami} returns a maximum value of 1, but can also drop below 0.\\
The user can add all scores to a comparison using a data structure called SessionState (see section \ref{implementation}). All added calculations are visualized in the form of an interactive histogram created with Altair (see section \ref{altair}). 