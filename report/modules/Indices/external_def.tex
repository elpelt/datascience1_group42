Validation of clustered data is a fundamental challenge in the clustering problem. To obtain a comparable and evaluable numerical value, validity indices are usually used. There are many different indices calculating this value in different ways. For example, external cluster validation scores. External criteria are defined as the evaluation of the result with respect to a given structure, here given as a classification in the original data. Thus, an external index compares the predefined cluster labels with the set of computed labels and returns a value based on the differences and similarities respectively. In short, it is based on previous knowledge of the data \cite{int_ext}.

All cluster algorithms can be inconsistent in naming clusters due to randomized initialisation. Thus, all indices require that a permutation of the cluster label values does not change the score value in any way, because the clustering remains the same. All four external indices explained below have this property.\\
These scoring methods were also chosen, because they are implemented in the scikit-learn library \cite{scikitlearn}, making it easy to include all of them.