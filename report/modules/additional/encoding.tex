Categorical data is represented by specific discrete values or labels. This is case for the housevotes dataset (see section \ref{housevotes}), where 
voting results can have one of three values (y, n, ?).
The distance measures described in section \ref{distances} need numerical data to work. The categorical data therefore needs to be converted (encoded) to numbers which accurately describe their distance to another.
Using simple integer encoding where no is encoded as 0, yes is encoded as 1 and the unknown state is encoded as 2 results in a yes vote being classified as closer to the unknown state than a no vote by the distance measures.\\
\ \\
For so called non ordinal data (data which has no known order) like the votes, One-Hot-Encoding provides better results when using distance based clustering.\\
With One-Hot-Encoding every attribute is represented as binary vector. Each element of this vector represents a category value. The corresponding value of a sample is set to 1 in the binary vector. 
This increases the dimensionality of the problem, but represents an equal distance between every value an attribute can have.\\
\ \\
The scikit-learn library \cite{scikitlearn} was used to implement One-Hot-Encoding on the housevotes dataset.
