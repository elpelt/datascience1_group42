For a variable $p \in \mathbb{N}$ the $L_p$\textit{-metrics} are defined as
\begin{align} \label{lpmetric}
    d(x, y) = \sum_{i=1}^{n}(|x_i-y_i|^p)^\frac{1}{p}
\end{align}
Setting $p = 2$ expresses the Euclidean distance, which is defined as the positive square root of the sum of all squared differences in each dimension:
\begin{align}
    d(x, y) = \sqrt{\sum_{i = 1}^{n}(|x_i - y_i|)^2}
\end{align}

The first two axioms defined in section \ref{def_DM} are easily shown to apply:
\begin{enumerate}

\item Identity of indiscernibles:\\
For $x = y$ the value is obviously $0$. Let $x = y$, then $(|x - y|)^2 = 0$ and $\sqrt{0} = 0$.

\item Symmetry: \\
Symmetry is cleary given by the square of each distance.\\
$(x - y)^2 = (y - x)^2$.
\end{enumerate}

Non-Negativity is also shown quite easily. The square of any real number is always positive and the squareroot of any real positive number is always positive. Hence $d(x, y) \geq 0$.\\
\ \\
The traingle inequality requires a more difficult proof. However, to keep it simple, the Euclidean space possesses the property that the sum of the lengths of Cathetus and Ancathetus is always longer than the length of the Hypothenuse \cite{MMDS}.