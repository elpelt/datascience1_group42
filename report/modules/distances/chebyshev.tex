\marginnote{\textcolor{cyan}{Niklas Conen}}
The Chebyshev distance (also known as Tschebyscheff distance, Maximum Value distance or $L_\infty$ distance) is the limit of the before mentioned $L_p$-metrics (equation \ref{lpmetric}). On a vector space this metric is induced by the Supremum Norm (also called Chebyshev Norm or Infinity Norm), which again is the limit of the $L_p$-norms.\\
Descriptively the Chebyshev metric is the greatest distance between two vectors on one axis. Formally it is defined as:
%
\begin{align}
    d(x, y) = \max (| x_i - y_i |)
\end{align}
%
which is the aforementioned limit of the $L_p$-metric and is therefore also called $L_\infty$-metric:
\begin{align}
    d(x, y) = \lim \limits_{p \to \infty} \left( \sum_{i=1}^{n}(|x_i-y_i|^p)^\frac{1}{p} \right)
\end{align}
\ \\
The three axioms for a metric (section \ref{metricaxioms}) are proven below:
\begin{enumerate}
    \item For $x = y$ all entries of a vector are identical and all differences between $x_i - y_i$ are $0$. Thus: $d(x,x) = \max (|x_i - x_i|) = \max (0) = 0$
    \item Symmetry is given because of the symmetry of the absolute value function: 
          $| x_i - y_i | = | y_i - x_i |$
    \item The triangle equation can be shown using some estimates:
    \begin{align*}
        \max(| x_i - y_i |) &= \max(|x_i - z_i + z_i - y_i|)\\
         &\leq \max(|x_i - z_i|+ |z_i - y_i|) \\
         &\leq \max(| x_i - z_i |) + \max(|z_i - y_i|)\\
        \Rightarrow d(x,y) &\leq d(x, z) + d(z, y)
    \end{align*}
\end{enumerate}
Non negativity also results from the non negativity of the absolute value function. Therefore the Chebyshev distance is classified as a metric.
