The web frontend is designed to provide an intuitive exploration of the different datasets, clustering algorithms and distances. In an interactive interface multiple clustering settings can be chosen and a visualisation of the results is directly generated. 
A cluster-table is used to store previous calculated results, which can be plotted within the evaluation module. \\
A checkbox at the top of the page, which is set by default, allows the use of precalculated clustering results, which have been computed beforehand (with a random seed value) and are stored in the github repository. This was done for reproduction purposes. \\
The user can choose between four datasets, four distance measures and four different algorithms. If kmeans, kmedian or kmedoids is chosen a value for the parameter k between 1 and 10 has to be set (default k=3). If DBSCAN is chosen the user has to define a value for epsilon and the minimal number of nearest points. The parameter settings can be adjusted with an interactive slider widget. \\
Moreover the perplexity value for the t-SNE projection can be set individually between 5 and 50 with a slider (default value is 25). The t-SNE and PCA plots are shown next to each other to allow a direct comparison of the lower-dimensional projections. \\
The frontend is reloaded entirely if a parameter value is changed, a different setting is made or a button is clicked. 
Ancillary streamlit settings can be found at the upper right corner. Additional explanatory texts provide help and more information. \\
To motivate the user and claim serious scientific standards some ballons have been added. 
