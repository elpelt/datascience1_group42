The Diabetes dataset contains various information as numeric values about 442 diabetic patients, namely age, sex, body mass index, average blood pressure, and six blood serum measurements (first 10 columns), as well as a quantitative measure of disease progression one year after baseline, i.e., the response of interest (11th column). All characteristics were standardized to standard deviation times $n$ samples and also mean centered.\\
This dataset is taken from the diabetes study conducted by Efron et al. \cite{diabetes} with the main goal of constructing a model that predicts the response (column 11) from the covariates (column 1-10).\\
Compared to all other datasets, there are numerical regression targets rather than categorical class values. Thus, there are no predefined clusters and thus no external cluster validation methods can be used for evaluation. Hence, the silhouette score is implemented and should be used for evaluation purposes.
