The Iris Dataset, introduced by Ronald Fisher in 1936, contains the petal and sepal measurements of three different species of Iris flowers \cite{fisher1936use}. The considered irises are Setosa, Versicolour and Virginica and each flower is represented in its own class. For each iris, 50 samples are included and for each sample the entirely numerical dataset contains the sepal length, sepal width, petal length and petal width in cm. Hence, the dataset consists of four columns where each represents one of the mentioned measurements and 150 lines, one for each individual flower. This dataset can be classified as multivatiate as there are four different features for each instance. \\
Although is dataset is comparatively small with only 150 instances in total, it is widely known and very popular for various statistical classification methods such as machine learning and support vector machines \cite{inproceedings}. Due to its popularity, it is not only included in the UCI Machine Learning Repository \cite{Dua2019}, but also in the base version of R \cite{rbase} and in the Python Scikit-learn package \cite{scikitlearn}.\\
While one species of iris is linerally seperable from the other irises, the other two are not linerally seperable from each other \cite{Dua2019}. This makes this dataset an interesting case for clustering.