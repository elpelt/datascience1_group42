\documentclass[12pt]%[english]
{article}

\usepackage{graphicx}
\usepackage{microtype}
\usepackage[nonumberlist, acronym, toc, section]{glossaries}
\usepackage{cite} 


\newglossary[slg]{symbolslist}{syi}{syg}{Symbols}

\makeglossaries

%commands for symbols
\newglossaryentry{symb:Pi}{
name=$\pi$,
description={You know it.},
sort=symbolpi, type=symbolslist
}
\newglossaryentry{symb:Phi}{
name=$\varphi$,
description={At vero eos et accusam et justo duo dolores et ea rebum..},
sort=symbolphi, type=symbolslist
}
\newglossaryentry{symb:Lambda}{
name=$\lambda$,
description={Lorem ipsum dolor sit amet, consetetur sadipscing elitr, sed diam nonumy.},
sort=symbollambda, type=symbolslist
}
 
%commands for abbreviations
\newacronym{MS}{MS}{Microsoft}
\newacronym{CD}{CD}{Compact Disc}

% ...

%abbreviations and glossary combined
\newacronym{AD}{AD}{Active Directory\protect\glsadd{glos:AD}}
 
% ... 
 
%glossary commands

\newglossaryentry{glos:AD}{
name=Active Directory,
description={At vero eos et accusam et justo duo dolores et ea rebum. Stet clita kasd gubergren, no sea takimata sanctus est Lorem ipsum dolor sit amet}
}

\newglossaryentry{glos:F}{name=File, description={An arbitrary file}}

% ...

\begin{document}

\begin{titlepage}

\begin{center}

{\Huge {
Term Paper Data Science 1}
}
\\[2ex]

\textbf{
\Large 
Docent: Prof. Dr. Lena Wiese \\ 
Semester: Summer Term 2021\\  
}



\includegraphics[scale=0.4]{logo.jpg} \\ 
\large{\textbf{Institute of Computer Science \\ Goethe-Universit\"at Frankfurt a. M.}}



\begin{tabular}{ll}
Authors: & \textsc{Franziska Hicking} \\
& {\small your Student ID} \\
& {\small your.email@ddre.ss} \\
& {\small branch of study (Bachelor/Master, semester count)} \\
& \textsc{Jonas Elpelt} \\
& {\small your Student ID}\\
&{\small  your.email@ddre.ss}\\
& {\small branch of study (Bachelor/Master, semester count)} \\
& \textsc{Julian Rummel} \\
&{\small  6673334}\\
& {\small s9594673@stud.uni-frankfurt.de}\\
&{\small  Master Bioinformatics, 2} \\
& \textsc{Niklas Conen}\\
& {\small your Student ID}\\
& {\small your.email@ddre.ss}\\
& {\small branch of study (Bachelor/Master, semester count)}\\
Date: & \today \\		
\end{tabular}

\end{center}

\vspace*{\fill}

\large
\noindent{}Chosen Project Topic: \\
T4 - DISTANCE MEASURES AND CLUSTERING


\end{titlepage}

\newpage\thispagestyle{empty}~ %empty page
\newpage 

\begin{abstract}
Lorem ipsum dolor sit amet, consetetur sadipscing elitr, sed diam nonumy eirmod tempor invidunt ut labore et dolore magna aliquyam erat, sed diam voluptua. At vero eos et accusam et justo duo dolores et ea rebum. Stet clita kasd gubergren, no sea takimata sanctus est Lorem ipsum dolor sit amet. Lorem ipsum dolor sit amet, consetetur sadipscing elitr, sed diam nonumy eirmod tempor invidunt ut labore et dolore magna aliquyam erat, sed diam voluptua. At vero eos et accusam et justo duo dolores et ea rebum. Stet clita kasd gubergren, no sea takimata sanctus est Lorem ipsum dolor sit amet. Lorem ipsum dolor sit amet, consetetur sadipscing elitr, sed diam nonumy eirmod tempor invidunt ut labore et dolore magna aliquyam erat, sed diam voluptua. At vero eos et accusam et justo duo dolores et ea rebum. Stet clita kasd gubergren, no sea takimata sanctus est Lorem ipsum dolor sit amet.   

Duis autem vel eum iriure dolor in hendrerit in vulputate velit esse molestie consequat, vel illum dolore eu feugiat nulla facilisis at vero eros et accumsan et iusto odio dignissim qui blandit praesent luptatum zzril delenit augue duis dolore te feugait nulla facilisi. Lorem ipsum dolor sit amet,
\end{abstract}

\newpage

\tableofcontents

\newpage

Some Latex-specific hints:


\begin{itemize}
\item You can use abbreviations, like \gls{AD}, \gls{MS} or \gls{CD}.
\item There are also symbols like for example \gls{symb:Pi}, \gls{symb:Phi} and \gls{symb:Lambda}.
\item Last but not least, use glossary entries like \gls{glos:AD} and \gls{glos:F}.
\item Do not forget to cite related work, like \cite{okman2011security} or \cite{borthakur2011apache}.
\end{itemize}

\section{Definition of Distance Measure}
\label{def_DM}
ANMERKUNGEN:
\begin{itemize}
\item What general problem is addressed?
\item What is the general methodology that is used?
\end{itemize}
A distance measure is a function $d(x, y)$ that calculates a real value between two points in a space, containing two sets of points. This function must satisfy the four following axioms:
\begin{enumerate}
\item No negative distances:\\
$d(x, y) \geq 0$
\item Identity of indiscernibles:\\
$d(x, y) = 0$, iff $x = y$
\item Symmetry: \\
$d(x, y) = d(y, x)$
\item Triangle inequality: \\
$d(x, y) \leq d(x, z) + d(z, y)$
\end{enumerate}
The triangle-inequality impose the condition that a distance reflects the shortest path between two points. Thus, it is not possible to achieve a distance improvement by traveling via an intermediate point $z$. \cite{MMDS}

\section{Different Distance Measurements}
ANMERKUNGEN:
\begin{itemize}
\item What specific problem is addressed?
\item What is the specific methodology that is used?
\item What improvement is shown?
\end{itemize}
\subsection{Euclidean Distances}
The Euclidean distance calculates the distance of two points, represented as vectors of $n$ real numbers, in an $n$-dimensional euclidean space. In general, a Euclidean distance measure $d$ is called $L_r$\textit{-norm}, for arbitrary constants r. 
$$d([x_1,x_2,...,x_n], [y_1,y_2,...,y_n])=\sum_{i=1}^{n}(|x_i-y_i|^r)^\frac{1}{r}$$
The typical euclidean norm refers to the $L_2$\textit{-norm} and is calculated as the positive square root of the sum of all squared distances in each dimension.
$$d([x_1,x_2,...,x_n], [y_1,y_2,...,y_n])=\sqrt{\sum_{i=1}^{n}(|x_i-y_i|)^2}$$
It is simple to verify three of the aforementioned axioms (\ref{def_DM}).
\begin{enumerate}
\item No negative distances:\\
The nonnegative values are given by the positive square root. let $x \neq y$, then the square of any real number is always positive.
\item Identity of indiscernibles:\\
For $x = y$ the value is obviously $0$. Let $x = y$, then $(|x - y|)^2 = 0$ and $\sqrt{0} = 0$.
\item Symmetry: \\
Symmetrie is cleary given by the square of each distance.\\
$(x - y)^2 = (y - x)^2$.
\item Triangle inequality: \\
As a matter of fact, this axiom requires a more difficult proof. However, to keep it simple, the Euclidean space possesses the property that the sum of the lengths of Cathetus and Ancathetus is always longer than the length of the Hypothenuse. \cite{MMDS}
\end{enumerate}

\section{Data Set Description}
ANMERKUNGEN:
\begin{itemize}
\item What benchmark data sets are used?
\end{itemize}

\section{Description of Python libraries used}
ANMERKUNGEN:
\begin{itemize}
\item Describe in detail what technology/library is used.
\end{itemize}

\section{Description of Evaluation Module}
ANMERKUNGEN:
\begin{itemize}
\item What are the results and how are they measured?
\end{itemize}

\section{Web Frontend and User Manual}
ANMERKUNGEN:
\begin{itemize}
\item Describe the implementation and write a brief user manual with screenshots.
\end{itemize}

\section{Conclusion}
ANMERKUNGEN:
\begin{itemize}
\item Summarize the main points and achievements
\item Add your own assessment/criticism on the topic
\end{itemize}

\newpage

%print glossary
\printglossary[style=altlist,title=Glossary]
 
%print abbreviations
\printglossary[type=\acronymtype,style=long]
 
%print symbols
\printglossary[type=symbolslist,style=long]

\newpage

\bibliography{bib}
\bibliographystyle{unsrt}

\end{document}
